% ---------------------------------
% -- Assignment 1: Chapter 1, #1 --
% -- Math 4171 --------------------

% Set document type
\documentclass[11pt,oneside]{extarticle}

\usepackage{amsmath}
\usepackage{amssymb}
\usepackage{amsthm}
\usepackage{garamondx}
\usepackage{graphicx}
\usepackage{hyperref}
\usepackage{newtxmath}

% Real number symbol
\newcommand{\Real}{\mathbb{R}}

% Change enumerator to use letters i.e., a, b, c, ...
\renewcommand{\theenumi}{\alph{enumi}}

\begin{document}
\section{Chapter 3}

\subsection{Exercise 1}

The \textsc{Rosenbrock} function

$$
f:\Real^2 \rightarrow \Real, x \mapsto 100(x_2-x_1^2)^2 + (1-x_1)^2,
$$

also compare \url{http://en.wikipedia.org/wiki/Rosenbrock_function}, is frequently
utilized to test optimization methods.

\begin{enumerate}

    \item The absolute minimum of $f$, at the points $(1,1)^T$, can be seen 
        without any calculations. Show that it is the only extremal point.

    All local (and thus global) extrama occur at critical points of $f$, therefore
    it is sufficient to show that the only critical point exists at $(1,1)^T$.

    $$\nabla f(x)=\begin{pmatrix}
        200(x_2 - x_x^2 ) (-2x_1) - 2(1-x_1) \\
        200(x_2-x_1^2) \\
    \end{pmatrix}$$

    The only solution to $\nabla f(x) = 0$ is $x=(1,1)^T$, which is at the
    absolute minimum $(1,1)^T$, which we already have. There are no other critical
    points, thefore this is the only extrema.

    \item Graph the function $f$ on $[-1.5,1.5]\times [-0.5,2]$ as a 3D plot or
        as a level curve plot to understand why $f$ is reffered to as the
        \emph{bananna function} and the like.

    \item Implement the \textsc{Nelder-Mead} method. Visualize the level curves of the
        given function together with the polytop for each iteration. Finnaly
        visualize the trajectory of the ecenters of gravity of the polytopes!

    \item Test the program with the starting polytope given by the vertices
        $(-1,1)^T$, $(0,1)^T$, $(-0.5,2)^T$ and the parameters
        $(\alpha,\beta,\gamma):=(1,2,0.5)$ and $\varepsilon := 10^{-4}$ using
        the \textsc{Rosenbrock} function. How many iterations are needed? What
        is the distance between the calculated solution and the exact minimizer
        $(1,1)^T$?

    \item Find $(\alpha, \beta, \gamma ) \in \left[ 0.9,1.1\right] \times
        \left[1.9,2.1\right] \times \left[0.4,0.6\right]$ such that with
        $\varepsilon := 10^{-4}$ the algorithm terminates after as few iterations
        as possible. What is the distance between the solution and the minimizer
        $(1,1)^T$ in this case?

    \item If the distance in \emph{e} was greater than in \emph{d}, reduce
        $\varepsilon$ until the algorithm gives a result---with the $(\alpha,\beta,\gamma)$,
        found in \emph{e}---which is not farther away from $(1,1)^T$ and at the same
        time needs fewer iterations than the solution in \emph{d}.

\end{enumerate}

\end{document}
